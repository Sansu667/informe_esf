\documentclass{article}\usepackage[]{graphicx}\usepackage[]{xcolor}
% maxwidth is the original width if it is less than linewidth
% otherwise use linewidth (to make sure the graphics do not exceed the margin)
\makeatletter
\def\maxwidth{ %
  \ifdim\Gin@nat@width>\linewidth
    \linewidth
  \else
    \Gin@nat@width
  \fi
}
\makeatother

\definecolor{fgcolor}{rgb}{0.345, 0.345, 0.345}
\newcommand{\hlnum}[1]{\textcolor[rgb]{0.686,0.059,0.569}{#1}}%
\newcommand{\hlsng}[1]{\textcolor[rgb]{0.192,0.494,0.8}{#1}}%
\newcommand{\hlcom}[1]{\textcolor[rgb]{0.678,0.584,0.686}{\textit{#1}}}%
\newcommand{\hlopt}[1]{\textcolor[rgb]{0,0,0}{#1}}%
\newcommand{\hldef}[1]{\textcolor[rgb]{0.345,0.345,0.345}{#1}}%
\newcommand{\hlkwa}[1]{\textcolor[rgb]{0.161,0.373,0.58}{\textbf{#1}}}%
\newcommand{\hlkwb}[1]{\textcolor[rgb]{0.69,0.353,0.396}{#1}}%
\newcommand{\hlkwc}[1]{\textcolor[rgb]{0.333,0.667,0.333}{#1}}%
\newcommand{\hlkwd}[1]{\textcolor[rgb]{0.737,0.353,0.396}{\textbf{#1}}}%
\let\hlipl\hlkwb

\usepackage{framed}
\makeatletter
\newenvironment{kframe}{%
 \def\at@end@of@kframe{}%
 \ifinner\ifhmode%
  \def\at@end@of@kframe{\end{minipage}}%
  \begin{minipage}{\columnwidth}%
 \fi\fi%
 \def\FrameCommand##1{\hskip\@totalleftmargin \hskip-\fboxsep
 \colorbox{shadecolor}{##1}\hskip-\fboxsep
     % There is no \\@totalrightmargin, so:
     \hskip-\linewidth \hskip-\@totalleftmargin \hskip\columnwidth}%
 \MakeFramed {\advance\hsize-\width
   \@totalleftmargin\z@ \linewidth\hsize
   \@setminipage}}%
 {\par\unskip\endMakeFramed%
 \at@end@of@kframe}
\makeatother

\definecolor{shadecolor}{rgb}{.97, .97, .97}
\definecolor{messagecolor}{rgb}{0, 0, 0}
\definecolor{warningcolor}{rgb}{1, 0, 1}
\definecolor{errorcolor}{rgb}{1, 0, 0}
\newenvironment{knitrout}{}{} % an empty environment to be redefined in TeX

\usepackage{alltt}
\usepackage[utf8]{inputenc}
\usepackage[margin=2.5cm]{geometry}
\usepackage{graphicx}
\usepackage{hyperref}
\usepackage{booktabs}
\usepackage{float}
\usepackage{amsmath}

\renewcommand{\tablename}{Tabla}
\renewcommand{\figurename}{Figura}
\renewcommand{\contentsname}{Contenido}

\title{\textbf{Análisis del desempeño de Bilibili Gaming en el World Championship 2024 de League of Legends}}
\author{Valentina Fonseca, Cristian Pérez, Edgar Santiago Suarez Alzate}
\date{\today}
\IfFileExists{upquote.sty}{\usepackage{upquote}}{}
\begin{document}

\maketitle





\begin{knitrout}
\definecolor{shadecolor}{rgb}{0.969, 0.969, 0.969}\color{fgcolor}\begin{kframe}
\begin{verbatim}
##  [1] "TeamName"      "PlayerName"    "Position"      "Games"        
##  [5] "Win_rate"      "KDA"           "Avg_kills"     "Avg_deaths"   
##  [9] "Avg_assists"   "CSPerMin"      "GoldPerMin"    "KPPercent"    
## [13] "DamagePercent" "DPM"           "VSPM"          "Avg_WPM"      
## [17] "Avg_WCPM"      "Avg_VWPM"      "GD_15"         "CSD_15"       
## [21] "XPD_15"        "FBPercent"     "FB_Victim"     "Penta_Kills"  
## [25] "Solo_Kills"    "Country"       "FlashKeybind"
\end{verbatim}
\end{kframe}
\end{knitrout}





\section{Introducción}

El World Championship 2024 de League of Legends representó uno de los torneos más competitivos de los esports a nivel mundial. Este informe se centra en analizar los factores que pudieron influir en que \textit{Bilibili Gaming} no lograra obtener el campeonato mundial. 

\section{Análisis descriptivo}

\subsection{Descripción general de los datos}
La base de datos analizada contiene información de \textbf{81} jugadores profesionales de \textbf{16} equipos y \textbf{27} variables. Pero antes de continuar es bueno explicar en qué consiste las 12 variables que son importantes para nuestro proposito:
\begin{enumerate}
  \item \textit{TeamName}: Nombre del equipo participante
  \item \textit{PlayerName}: Identificador único del jugador
  \item \textit{Position}: Rol (Top, Jungle, Mid, Adc, Support)
  \item \textit{Games}: Total de partidas disputadas
  \item \textit{Win\_rate}: Tasa de victorias (0-1)
  \item \textit{KDA}: Ratio (Kills + Assists) / Deaths 
  \item \textit{Avg\_kills}: Promedio de eliminaciones 
  \item \textit{Avg\_deaths}: Promedio de muertes 
  \item \textit{Avg\_assists}: Promedio de asistencias 
  \item \textit{GoldPerMin}: Oro por minuto
  \item \textit{DamagePercent}: \% del daño del equipo
  \item \textit{Country}: País de origen
\end{enumerate}

\subsection{Estadísticas descriptivas por equipo}
\begin{knitrout}
\definecolor{shadecolor}{rgb}{0.969, 0.969, 0.969}\color{fgcolor}\begin{table}[!h]
\centering
\caption{\label{tab:estadisticas_equipos}Top 10 equipos por tasa de victoria}
\centering
\resizebox{\ifdim\width>\linewidth\linewidth\else\width\fi}{!}{
\begin{tabular}[t]{lrrrrr}
\toprule
Equipo & N. de Jugadores & Win Rate & KDA & GPM & Dmg \%\\
\midrule
T1 & 5 & 0.88 & 7.06 & 391.8 & 0.20\\
Gen.G & 5 & 0.70 & 5.94 & 376.4 & 0.20\\
Bilibili Gaming & 6 & 0.63 & 4.12 & 370.5 & 0.19\\
LNG Esports & 5 & 0.62 & 5.42 & 379.2 & 0.20\\
Weibo Gaming & 5 & 0.62 & 4.70 & 372.8 & 0.20\\
\addlinespace
FlyQuest & 5 & 0.50 & 3.94 & 359.4 & 0.20\\
Hanwha Life Esports & 5 & 0.50 & 3.50 & 366.8 & 0.20\\
Team Liquid & 5 & 0.50 & 3.44 & 370.4 & 0.20\\
Top Esports & 5 & 0.50 & 3.52 & 356.6 & 0.20\\
G2 Esports & 5 & 0.38 & 3.34 & 370.6 & 0.20\\
\bottomrule
\end{tabular}}
\end{table}

\end{knitrout}

\subsection{Análisis de Bilibili Gaming}
\begin{table}[!h]
\centering
\caption{\label{tab:stats_bilibili}Estadísticas de Bilibili Gaming}
\centering
\begin{tabular}[t]{rrrrrr}
\toprule
Jugadores & WinRate & KDA & Kills & Deaths & GPM\\
\midrule
6 & 0.63 & 4.12 & 3.03 & 3.05 & 370.5\\
\bottomrule
\end{tabular}
\end{table}



\subsection{Análisis por posición}
\begin{knitrout}
\definecolor{shadecolor}{rgb}{0.969, 0.969, 0.969}\color{fgcolor}\begin{figure}[H]

{\centering \includegraphics[width=\maxwidth]{figure/grafico_posiciones-1} 

}

\caption[KDA por posición]{KDA por posición}\label{fig:grafico_posiciones}
\end{figure}

\end{knitrout}

El diagrama box plot evidencia diferencias en el rendimiento entre posiciones. Los jugadores de Mid y Adc presentan los valores medianos de KDA más altos, indicando una mayor eficiencia en eliminaciones y asistencias respecto a muertes. En contraste, Top y Support muestran desempeños más modestos, aunque con menor dispersión, lo que sugiere una función más estable y menos dependiente de las estadísticas ofensivas.

\begin{knitrout}
\definecolor{shadecolor}{rgb}{0.969, 0.969, 0.969}\color{fgcolor}\begin{figure}[H]

{\centering \includegraphics[width=\maxwidth]{figure/grafico_winrate_gpm-1} 

}

\caption[Win Rate vs Gold Per Min]{Win Rate vs Gold Per Min}\label{fig:grafico_winrate_gpm}
\end{figure}

\end{knitrout}

Se aprecia una asociación positiva débil entre el oro por minuto y la tasa de victorias: la línea de tendencia es ascendente pero de baja pendiente. No obstante, existen excepciones notables (por ejemplo, varios Supports con bajo GPM y alto Win Rate) y la relación varía según la posición. Estas observaciones indican que, si bien la eficiencia económica contribuye al éxito, no explica por sí sola las victorias y su efecto es heterogéneo entre roles.

\section{Análisis inferencial}

\subsection{Diagnóstico visual de distribuciones}
Para validar la robustez de nuestros análisis, inspeccionamos visualmente la distribución de las variables críticas (\textit{KDA} y \textit{GoldPerMin}) comparándolas con una curva normal teórica.

\begin{knitrout}
\definecolor{shadecolor}{rgb}{0.969, 0.969, 0.969}\color{fgcolor}\begin{figure}[H]

{\centering \includegraphics[width=\maxwidth]{figure/distribucion_visual-1} 

}

\caption[Distribución de KDA y Gold Per Min con Curva Normal]{Distribución de KDA y Gold Per Min con Curva Normal}\label{fig:distribucion_visual}
\end{figure}

\end{knitrout}

Visualmente, el \textit{KDA} muestra una ligera asimetría positiva (cola a la derecha), común en métricas de rendimiento donde unos pocos jugadores destacan mucho. El \textit{GoldPerMin} parece ajustarse mejor a la normalidad.

\subsection{Análisis de eficiencia de recursos}
Introducimos una nueva métrica: \textbf{eficiencia de recursos} ($E_R$). A partir de ahora utilizaremos este factor como \textbf{Efficiency}.
$$ E_R = \frac{\text{DamagePercent}}{\text{GoldPerMin}} \times 1000 $$
Esta métrica cuantifica cuánto daño aporta un jugador por cada unidad de oro que consume. Un valor alto indica un jugador que hace "más con menos", optimizando la economía del equipo.

\begin{itemize}
\item \textbf{Eficiencia promedio global:} 0.52
\item \textbf{Eficiencia promedio BLG:} 0.5
\item \textbf{P-valor de la diferencia:} 0.717
\end{itemize}

A pesar de su éxito, la eficiencia de recursos de BLG no difiere significativamente del promedio. Su ventaja podría radicar en la obtención bruta de recursos más que en la eficiencia de su uso.

\subsection{Intervalos de confianza}
Para ubicar el desempeño de Bilibili Gaming en el contexto global, calculamos los \textbf{intervalos de confianza del 95\%} para la media de cada métrica.
\begin{itemize}
    \item Las \textbf{barras negras} representan el rango donde se encuentra el verdadero promedio global con un 95\% de confianza.
    \item El \textbf{punto rojo} indica el valor observado de Bilibili Gaming.
\end{itemize}
Si el punto rojo cae fuera de las barras, podemos afirmar con un 95\% de seguridad que BLG es significativamente diferente al promedio. Si cae dentro, su desempeño es estadísticamente normal.

\begin{knitrout}
\definecolor{shadecolor}{rgb}{0.969, 0.969, 0.969}\color{fgcolor}\begin{figure}[H]

{\centering \includegraphics[width=\maxwidth]{figure/ci_visual-1} 

}

\caption[Comparación de BLG vs Intervalos de confianza globales (95\%)]{Comparación de BLG vs Intervalos de confianza globales (95\%)}\label{fig:ci_visual}
\end{figure}

\end{knitrout}

\subsection{Regresión lineal y diagnóstico de residuos}
Ajustamos un modelo de regresión lineal múltiple para cuantificar qué factores determinan el éxito (\textit{Win Rate}). El modelo propuesto es:
$$ \text{Win\_rate} = \beta_0 + \beta_1 \cdot \text{KDA} + \beta_2 \cdot \text{GoldPerMin} + \beta_3 \cdot \text{Efficiency} + \epsilon $$

\begin{table}[!h]
\centering
\caption{\label{tab:regresion_diagnostico}Coeficientes del Modelo Final}
\centering
\begin{tabular}[t]{lrrrl}
\toprule
  & Estimado & Error\_Std & P\_Valor & Signif\\
\midrule
(Intercept) & 0.1239 & 0.0857 & 0.1521 & \\
KDA & 0.0671 & 0.0106 & 0.0000 & *\\
GoldPerMin & 0.0016 & 0.0005 & 0.0031 & *\\
Efficiency & -0.9631 & 0.2289 & 0.0001 & *\\
\bottomrule
\end{tabular}
\end{table}

\begin{figure}[H]

{\centering \includegraphics[width=\maxwidth]{figure/regresion_diagnostico-1} 

}

\caption[Diagnóstico del modelo]{Diagnóstico del modelo: Residuos y normalidad}\label{fig:regresion_diagnostico}
\end{figure}


El modelo presenta un $R^2$ ajustado de 0.58. Esto significa que el 58\% de la variabilidad en la tasa de victorias se explica por el KDA, el Oro y la Eficiencia.

\begin{itemize}
    \item \textbf{KDA (Positivo)}: Es el predictor más fuerte. Un mayor KDA está directamente asociado con un mayor Win Rate.
    \item \textbf{GoldPerMin (Positivo)}: Acumular oro también incrementa la probabilidad de victoria, aunque su impacto es menor que el del KDA.
    \item \textbf{Efficiency (Negativo)}: Curiosamente, el coeficiente es negativo. Esto podría deberse a la colinealidad con el Oro, o indicar que equipos que hacen "mucho daño con poco oro" (alta eficiencia) suelen ser equipos que van perdiendo y pelean desde atrás, mientras que los ganadores acumulan tanto oro que su eficiencia nominal baja.
\end{itemize}

Ahora es necesario que expliquemos qué quiere decir aquellos gráficos de la figura 5. El gráfico izquierdo nos muestra que la dispersión es aleatoria alrededor de 0, lo que valida la linealidad del modelo. El gráfico derecho nos muestra que los residuos siguen una distribución normal, lo que valida las pruebas de significancia del modelo.

\subsection{Análisis de varianza (ANOVA)}
Finalmente, realizamos una prueba ANOVA para verificar si la posición del jugador influye significativamente en la obtención de oro (\textit{GoldPerMin}).
\begin{itemize}
    \item $H_0$: El promedio de oro es igual para todas las posiciones.
    \item $H_1$: Al menos una posición obtiene un promedio de oro diferente.
\end{itemize}

\begin{table}[H]
\centering
\caption{\label{tab:anova_test}Tabla ANOVA: Oro por Posición}
\centering
\begin{tabular}[t]{lrrrll}
\toprule
Fuente & Df & Sum Sq & Mean Sq & F Value & P Value\\
\midrule
Posición & 4 & 385480 & 96370 & 241.37 & $<$ 1e-04\\
Residuales & 76 & 30344 & 399 &  & \\
\bottomrule
\end{tabular}
\end{table}


El valor p es extremadamente bajo ($< 0.05$), por lo que rechazamos la hipótesis nula. Esto confirma estadísticamente que \textbf{la posición determina la economía}: los carries (ADC, Mid) ganan significativamente más oro que los supports, lo cual es consistente con la estructura del juego.

\end{document}
